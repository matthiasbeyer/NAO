\chapter{Choreographe}

    Nachdem der Choreographe installiert wurde (\ref{chap:inst:choero}) kann
    dieser benutzt werden um sich mit dem Roboter zu verbinden.
    Dazu muss im Menü "`Connection'' die Auswahl "`Connect to..'' getroffen
    und der Robotername ausgewählt werden (\autoref{img:nao:choreo:welcome}).
    Dabei ist zu beachten dass die \ac{IP}-Adresse und der Port korrekt sind.

    Zur Entwicklung von Behaviour kann zudem der virtuelle Roboter der
    Software verwendet werden, es ist kein physisch vorhandener Roboter nötig.
    Dazu ist die Software "`naoqi'' nötig, welche mit der Installations-\ac{CD}
    beliegt.
    Sobald "`naoqi'' gestartet ist, wird im "`Connection''-Menü des Choreographe
    ein Eintrag zu ersterer sichtbar.

    Zudem kann, um reale Bedingungen zu simulieren, "`WebotsForNAO'' von der
    Installations-\ac{CD} installiert werden.

    Beiliegend zum Choreographe wird ein Programm "`Monitor'' geliefert, welches
    dazu dient, die Speichernutzung sowie die Kameras des NAO zu überwachen und
    verschiedene Optionen letzterer zu bearbeiten.

    % \section{Erstellen von Behaviour}

    Auf der linken Seite des Choreographe befinden sich die "`Box libraries'',
    welche benutzt werden können um einzelne Bewegungen zu komplexen
    zusammenzusetzen.
    Hier sind zudem Beispielanimationen und Sensorikfunktionalität auffindbar
    (\autoref{img:nao:choreo:boxlib}).

    Mittels "`drag-and-drop'' können einzelne Elemente auf der Arbeitsfläche
    positioniert werden (\autoref{img:nao:choreo:draganddrop}).
    Danach können die Input- und Output-Ports der einzelnen Behaviour-Module
    verbunden werden (\autoref{img:nao:choreo:connect}).
    Die Parameter und Optionen der Behaviour können bearbeitet werden, indem im
    unteren, linken Bereich der Behaviour das Optionen-Symbol betätigt wird
    (\autoref{img:nao:choreo:options}).
    In dem erscheinenden Dialog gibt es die Möglichkeit direkt Python-Code
    einzufügen um das Behaviour zu modifizieren\footnote{
        \ac{API}-Dokumentation unter \cite{nao:api}.
    }.


