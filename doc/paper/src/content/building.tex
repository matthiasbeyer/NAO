\chapter{Building}

    In diesem Kapitel soll beschrieben werden wie der Quellcode des
    Vorgängerprojektes gebaut und getested wurde.

    \section{Ubuntu}

        Sofern die Installation der Toolchain, ``OpenCV'' und ``qibuild''
        erfolgreich abgeschlossen wurde, kann nun das Quellcodeprojekt gebaut
        werden.

        Dazu wird der Quellcode im Arbeitsverzeichnis abgelegt und ``qibuild''
        entsprechend konfiguriert und übersetzt,
        anschließend wird das Projekt mit ``QtCreator'' geöffnet und dieser
        Konfiguriert (\ref{lst:buildproject}).

        \defaultlst{src/sources/build-project.sh}
                    {bash}
                    {Build Project}
                    {lst:buildproject}

        ``QtCreator'' wird als ``BuildPath'' folgender Pfad übergeben;

\begin{lstlisting}[caption={QtCreator build path},label=lst:qtcbp,language=bash]
~/nao/workspace/tictactoe/build-"toolchain-name"/sdk/bin/
\end{lstlisting}

        Das Projekt ist nun übersetzt und bereit editiert zu werden.

% vim: spelllang=de
