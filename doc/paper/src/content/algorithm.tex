\chapter{Spiele-Algorithmus}

    Der Algorithmus welcher den Ablauf des Spiels implementiert ist in
    \autoref{img:algo:sequence} aufgezeigt.

    Der Algorithmus bedient sich dabei folgenden einfachen Fakten:

    \begin{itemize}
        \item Es wird nicht mehr gezogen wenn\dots
            \begin{itemize}
                \item die $21$ mit weniger als der Hälfte der möglichen Karten
                    erreicht werden kann
            \end{itemize}
        \item Es wird gezogen wenn\dots
            \begin{itemize}
                \item die $21$ mit mehr als der Hälfte der möglichen Karten
                    erreicht werden kann
                \item Noch nicht gezogen wurde
            \end{itemize}
    \end{itemize}

    Das Softwaremodul welches den Algorithmus implementiert hat dabei eine
    wohldefinierte Schnittstelle, welche nach außen verfügbar ist.
    Der Algorithmus teilt der aufrufenden Entität eine ``Sicherheit'' mit,
    wie sicher sich der Algorithmus beim nächsten Zug ist.

    \begin{equation}
        \label{eq:algo:answer:no}
        A_{nein} = \{ a \mid a < 0, a >= -100 \}
    \end{equation}

    Diese ``Sicherheit'' wird durch eine Ganzzahl im Bereich $-100..100$
    dargestellt, wobei eine negative Sicherheit ein ``Nein''
    (\ref{eq:algo:answer:no}) angibt, eine positive ein ``Ja''
    (\ref{eq:algo:answer:yes}) auf die Frage, ob noch einmal gezogen werden soll.

    \begin{equation}
        \label{eq:algo:answer:yes}
        A_{ja} = \{ a \mid a >= 0, a <= 100 \}
    \end{equation}

    Die Klasse, welche die Antwort auf diese Frage darstellt (``Draw'') stellt
    entsprechende Methoden zur Umwandlung in Wahrheitswerte und Prozentzahl
    zur Verfügung.

    Ein Benutzer der Schnittstelle hat verschiedene Usecases, welche in
    \autoref{img:algo:usecases} visualisiert sind.

    Das resultierende Klassendiagramm (\autoref{img:algo:classes}) zeigt die
    Algorithmus-Komponente.
    Es existieren drei Helfer-Klassen, welche in Folgendem beschrieben werden
    sollen.

    \section{Helferklasse: Card}

        Diese Klasse repräsentiert eine ``Karte'' im Spiel.
        Sie existiert um eine Karte zu abstrahieren und stellt ein einfaches
        Interface für den Algorithmus dar, das Konzept einer "`Karte'' zu
        benutzen.

    \section{Helferklasse: State}

        Diese Klasse (eine Enumeration) stellt den aktuellen Status des
        Algorithmus dar.
        Der Algorithmus kann in verschiedenen Zuständen sein, wie zum Beispiel
        ``Spielend'' oder ``Angehalten''.

    \section{Helferklasse: Draw}

        Diese Klasse (eine Enumeration) stellt die Antwort auf die Frage, ob
        noch einmal gezogen werden soll, dar.
        Ein einfacher Wahrheitswert wäre zwar ausreichend, beschreibt allerdings
        nur ob gezogen werden soll oder nicht.
        Diese Klasse bietet eine abstraktere Antwort, damit der Roboter
        entsprechend verdeutlichen kann, wie sicher er sich ist beim nächsten
        Zug.

% vim: spelllang=de
