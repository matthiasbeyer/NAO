\chapter{Installation}

    Folgendes Kapitel beschreibt die Installation der Entwicklungsumgebung mitsamt
    Toolchain und ``qibuild''.

    \section{Ubuntu 12.04}

        In folgendem Kapitel wird eine virtuelle Maschine mit Ubuntu
        verwendet.
        Es wurde die Ubuntuversion aus Listing \ref{lst:ubuntuvers} verwendet.

        \defaultlst{src/sources/ubuntu-os-release.sh}
                    {bash}
                    {Ubuntu version}
                    {lst:ubuntuvers}

        Im Ersten Schritt werden alle benötigten Pakete installiert (siehe
        Listing \ref{lst:aptget}).
        Dazu wird das Systemtool ``apt-get'' verwendet.

        \defaultlst{src/sources/installation-packages.sh}
                    {bash}
                    {Installation der Pakete}
                    {lst:aptget}
        (entnommen aus \cite[S. 5, f.]{projss15})

        Zur Verwendung der Toolchain wird im Folgenden eine Ordnerstruktur
        angelegt, in welchem der Quellcode verwaltet wird (Listing
        \ref{lst:mkdirp}).

        \defaultlst{src/sources/folder-structure.sh}
                    {bash}
                    {Anlegen der Ordnerstruktur}
                    {lst:mkdirp}
        (entnommen aus \cite[S. 4]{projss15})

        Die Programmierumgebung ``qibuild'' welche die Toolchain zur Verfügung
        stellt muss installiert und konfiguriert werden (Listing
        \ref{lst:installqibuild}).

        \defaultlst{src/sources/install-qibuild.sh}
                    {bash}
                    {Installieren von qibuild}
                    {lst:installqibuild}
        (teilweise entnommen aus \cite[S. 5]{projss15})

        Nachdem die Umgebung erfolgreich installiert wurde, kann eine Toolchain
        angelegt werden (Listing \ref{lst:mktoolchain} und ``qibuild''
        initialisiert werden (Listing \ref{lst:initqibuild}).

        \defaultlst{src/sources/mk-toolchain.sh}
                    {bash}
                    {Erstellen der Toolchain}
                    {lst:mktoolchain}
        (entnommen aus \cite[S. 7]{projss15})

        Zudem muss OpenCV aus dem Quellcode installiert werden.
        Hierzu muss der OpenCV-Quellcode als ZIP-Archiv von der Herstellerseite
        heruntergeladen und wie in Listing \ref{lst:buildinstallopencv}
        dargestellt installiert werden:

        \defaultlst{src/sources/install-opencv.sh}
                    {bash}
                    {Build, Install: opencv}
                    {lst:buildinstallopencv}

        Die Linker-Konfiguration muss während des Installationsprozesses
        bearbeitet werden.
        In Listing \ref{lst:ldconfig} ist die vollständige Datei gelistet.

        \defaultlst{src/sources/ldconfig-change}
                    {bash}
                    {/etc/ld.so.conf}
                    {lst:ldconfig}


        \defaultlst{src/sources/qibuild-init.sh}
                    {bash}
                    {Init: qibuild}
                    {lst:initqibuild}
        (entnommen aus \cite[S. 8, f.]{projss15})

        Die Konfigurationsdatei für ``qibuild''
        (im Pfad ~/.config/qi/qibuild.xml) muss bearbeitet werden, die
        Toolchain muss eingetragen werden (Listing \ref{lst:qibuildxml}).

        \defaultlst{src/sources/qibuild.xml}
                    {xml}
                    {qibuild XML file}
                    {lst:qibuildxml}
        (entnommen aus \cite[S. 8]{projss15})

    \section{Installation unter Windows}

        Folgendes Kapitel beschreibt die Installation des NAO-SDKs unter
        Windows.

        \subsection{Systemvorraussetzungen}

            Die im Folgenden beschriebenen Schritte wurden auf einem Computer
            mit folgender vorinstallierter Software ausgeführt:

            \begin{itemize}
                \item Windows 10 Pro 64 Bit
                \item Python 2.7 (python-2.7.10.msi)
                \item CMake 2.8.3 (cmake-2.8.3-win32-x86.exe)
                \item Naoqi SDK 1.14.5 (naoqi-sdk-1.14.5-win32-vs2010.zip)
                \item qiBuild 1.14.3 (qibuild-1.14.3.zip)
                \item Microsoft Visual Studio 2010 Professional x86
            \end{itemize}

        \subsection{C++ SDK Installation unter Windows}

            Folgende Schritte müssen zur Installation des C++ SDKs unter Windows
            ausgeführt werden:

            \begin{itemize}
                \item Python 2.7 und CMake installieren
                \begin{itemize}
                    \item python-2.7.10.msi und cmake-3.3.2-win32-x86.exe
                        ausführen

                    \item Systemumgebungsvariablen öffnen und dabei
                        Systemvariablen ``Neu'' auswählen
                        (\autoref{img:win:sysvar1}, \autoref{img:win:sysvar2},
                        \autoref{img:win:sysvar3})

                    \begin{itemize}
                        \item ``Name der Variablen:'' ``PY\_HOME''
                            (\autoref{img:py:home})

                        \item ``Wert der Variablen:'' ``Python\_Pfad'',
                            zum Beispiel ``C:\textbackslash{}Python27'')
                    \end{itemize}

                    \item Bereits bestehende Variable ``Path'' bearbeiten und
                        Einträge hinzufügen, wie in Listing \ref{lst:win:path}
                        zu sehen.
                \end{itemize}

                \item ``quiBuild'' installieren
                \begin{itemize}
                    \item qibuild-1.14.3.zip nach „C:\textbackslash{}naoqi\textbackslash{}qibuild\textbackslash{}“ entpacken

                    \item C:\textbackslash{}naoqi\textbackslash{}qibuild\textbackslash{}install-qibuild.sh“ ausführen

                    \item Per Windows-Taste + $x$ $\rightarrow$ a und mit
                        ``Ja'' bestätigen - Administrator Eingabeaufforderung
                        aufrufen

                    \item Konfigurieren von qiBuild mittels
                        ``qibuild config --wizard'', dabei
                        ``10 Visual Studio 10 2010'' $\rightarrow$ ``7 Visual
                        Studio'' mit ``7'' bestätigen
                \end{itemize}

                \item naoqi-SDK installieren

                \begin{itemize}
                    \item naoqi-sdk-1.14.5-win32-vs2010.zip nach
                        ``C:\textbackslash{}naoqi\textbackslash{}sdk\textbackslash{}''
                        entpacken (Ggf. Ordernamen anpassen)

                    \item Wie zuvor die Administrator Eingabeaufforderungen
                        öffnen und mittels
                        ``cd C:\textbackslash{}naoqi\textbackslash{}''
                        zu diesem Ordner navigieren

                    \item„``qibuild init'' eingeben und per Enter bestätigen
                \end{itemize}

                \item Ein Beispielprogramm compilieren

                \begin{itemize}
                    \item Wie zuvor die Administrator Eingabeaufforderungen
                        öffnen und per
                        ``cd C:\textbackslash{}naoqi\textbackslash{}sdk\textbackslash{}doc\textbackslash{}examples''
                        zu diesem Ordner navigieren

                    \item Toolchain erstellen mit
                        ``„qitoolchain create mytoolchain C:\textbackslash{}naoqi\textbackslash{}sdk\textbackslash{}toolchain.xml''
                        (``mytoolchain'' entspricht dem Namen)

                    \item Um die Konfiguration zu erzeugen und anschließen das
                        Projekt zu bauen werden folgende Befehle ausgeführt:
                    \begin{itemize}
                        \item ``cd core/sayhelloworld'' - zum Projektorder
                            wechseln

                        \item ``qibuild configure -c mytoolchain'' - die
                            Konfiguration erzeugen

                        \item `„qibuild make -c mytoolchain'' - das Projekt
                            bauen
                    \end{itemize}

                \end{itemize}

            \end{itemize}

            Sofern keine Fehler aufgetreten sind, ist das SDK nun erfolgreich
            installiert und einsatzbereit.

            \lstinputlisting[language=bash,
                            caption={Windows Path},
                             label=lst:win:path]
                             {src/sources/windows-path-edit.txt}

        \subsection{OpenCV Installation unter Windows}

            Zur Installation von OpenCV unter Windows müssen folgende Schritte
            erfolgreich durchgeführt werden:

            \begin{itemize}
                \item "`opencv-2.4.11.exe'' ausführen und nach
                    "`C:\textbackslash{}naoqi\textbackslash{}'' installieren

                \item Systemumgebungsvariablen öffnen und die bereits bestehende
                    Variable "`Path'' bearbeiten, folgendes hinzufügen:
                    "`;C:\textbackslash{}naoqi\textbackslash{}opencv2\textbackslash{}build\textbackslash{}x86\textbackslash{}vc10\textbackslash{}bin;''

                \item "`OpenCV\_Debug.props'' und "`OpenCV\_Release.props'' nach
                    "`C:\textbackslash{}naoqi\textbackslash{}opencv\textbackslash{}props\textbackslash{}'' kopieren
            \end{itemize}

            Um ein NAOqi-Projekt in Microsoft Visual Studio 2010 zu erstellen,
            müssen folgende Schritte durchgeführt werden:

            \begin{itemize}
                \item per Windows-Taste $x \rightarrow a$ die
                    Administrator-Eingabeaufforderung aufrufen

                \item Mit "`cd C:\textbackslash{}naoqi\textbackslash{}\_workspace\textbackslash{}''
                    zu dem Arbeitsordner wechseln (Falls nicht vorhanden, diesen
                    Ordner erstellen)

                \item Ein neuen Worktree von qiBuild mit "`qibuild init -f''
                    anlegen

                \item Neues Toolchain für den Workspace  mit
                    "`qitoolchain create workspacechain C:\textbackslash{}naoqi\textbackslash{}sdk\textbackslash{}toolchain.xml''
                    erstellen

                \item Durch "`qibuild create meinProjekt'' ein neues Projekt
                    erstellen (Name frei wählbar).
                    Zur Inbetriebnahme eines bestehenden Projektes ab hier
                    einsteigen! (Nächster Punkt muss nicht immer beachtet werden)

                \item Anschließend
                    "`C:\textbackslash{}naoqi\textbackslash{}\_workspace\textbackslash{}meinProjekt\textbackslash{}CMakeLists.txt'' öffnen
                    und folgende Zeile in der letzten Zeile einfügen:
                    "`qi\_use\_lib(meinProjekt ALCOMMON)''
                    (Wichtig: Eigenen Projektnamen einsetzen)

                \item Die Visualstudio Projektdateien mit CMake generieren \\
                    "`cd C:\textbackslash{}naoqi\textbackslash{}\_workspace\textbackslash{}meinProjekt\textbackslash{}''
                    $\rightarrow$
                    "`qibuild configure -c workspacechain''


                \item Nun kann
                    "`C:\textbackslash{}naoqi\textbackslash{}\_workspace\textbackslash{}meinProjekt\textbackslash{}build-workspacechain\textbackslash{}meinProjekt.sln''
                    mit Microsoft Visual Studio 2010 geöffnet werden

                \item Im Projektmappen-Explorer auf "`meinProjekt'' ein
                    Rechtsklick ausführen $\rightarrow$
                    "`Als Startprojekt festlegen'' auswählen

            \end{itemize}

            Um OpenCV in Microsoft Visual Studio 2010 zu integrieren müssen
            folgende Schritte durchgeführt werden:

            \begin{itemize}
                \item Projekt öffnen
                \item Eigenschaftsmanager öffnen
                \item Zu "`meinProjekt'' $\rightarrow$ "`Debug Win32''
                    navigieren und anwählen
                \item "`Vorhandenes Eigenschaftsblatt hinzufügen'' aufwählen und
                    zu
                    "`C:\textbackslash{}naoqi\textbackslash{}opencv\textbackslash{}props''
                    navigieren. "`OpenCV\_Release.props'' auswählen

                \item Zum "`Projektmappen-Explorer'' wechseln und auf
                    "`meinProjekt'' rechtsklicken und "`Eigenschaften''
                    auswählen

                \item Nun zu "`Konfigurationseigenschaften'' $\rightarrow$ "`Linker'' $\rightarrow$
                    "`Eingabe'' navigieren und im rechten Feld bei "`zusätzliche
                    Abhängigkeiten'' auf den kleinen Pfeil nach unten klicken und
                    "`Bearbeiten'' auswählen. Nun den Hacken bei "`vom
                    übergeordneten Projekt erben oder ..'' setzen und mit "`OK''
                    und "`Übernehmen'' bestätigen
            \end{itemize}

            Nun kann mit Microsoft Visual Studio 2010 ein Programm mit NAOqi und
            OpenCV entwickelt werden.

    \section{Choreographe und Tools}
        \label{chap:inst:choero}

        Um den NAO Roboter mit Behaviours ausstatten zu können wird die
        "`Choreographe''-Software von Aldebaran verwendet.

        Um diese Installieren zu können wird die offizielle NAO-\ac{CD}
        benötigt, sowie ein Lizenzschlüssel, falls beabsichtigt wird die
        Software für mehr als 90 Tage zu benutzen.

% vim: spelllang=de
