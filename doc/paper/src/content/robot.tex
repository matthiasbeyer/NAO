\chapter{Der NAO-Roboter}

    Der NAO verfügt über verschiedenste Sensoren und Motoren.
    Er kann dank seiner Sensoren (Beschleunigungssensoren, Gyroskop)
    (\autoref{img:nao:sensors}) das Gleichgewicht
    beim Laufen wie auch bei verschiedenen Bewegungen beibehalten.
    Dank seiner Touch-Sensoren an Hand, Kopf und Fuß, sowie Sonarsensor kann der
    NAO seine Umgebung wahrnehmen und mit ihr interagieren.
    Die Interaktion zwischen Mensch und Roboter wird über Lautsprecher und
    Mikrofone im Kopf des NAO ermöglicht.
    Die Mikrofone können zudem für Spracherkennung genutzt werden, um zum
    Beispiel eine Sprachsteuerung des Roboters umzusetzen.
    Der Kopf des Roboters ist mit zwei hochauflösenden Kameras ausgestattet,
    welche Bild- und Echtzeitaufnahmen ermöglichen.
    Dabei gibt es eine Möglichkeit, Licht- und Farbverhältnisse automatisch
    optimieren zu lassen.

    \section{Inbetriebnahme}

        Im ausgeschaltenen Zustand hat der Roboter keine Körperspannung
        ("`Stiffness''), weswegen er mit Vorsicht behandelt werden muss.
        Insbesondere soll sein Kopf in einer natürlichen Position gehalten
        werden.

        Nachdem der Roboter aus der Packung entpackt ist und in eine aufrechte
        Position gebracht wurde (\autoref{img:nao:sitting}), kann er, indem 3-4
        Sekunden auf den Knopf welcher sich im Kopf des NAO befindet, gestartet
        werden.
        Die Computer des NAO booten nun und die verschiedenen Körperteile des
        NAO leuchten in verschiedenen Farben\footnote{
            Da der NAO nach dem Einschalten eventuell ein Standard-Behaviour
            startet, sollte er während seiner Initialisierungsphase unter
            Aufsicht bleiben.
        }.

        Der NAO verfügt über einen Lithium-Akku welcher eine Selbstständigkeit
        des Roboters für etwa 60 Minuten gewährleisten kann.
        Der NAO macht selbst auf einen niedrigen Batteriestand mittels einer
        Sprachnachricht aufmerksam. Der NAO kann auch mit Netzstrom verwendet
        werden.

        Der Roboter überwacht selbstständig die Temperatur in seinem Inneren und
        warnt den Nutzer mit einee Sprachnachricht über zu hohe Temperaturen.
        Erreicht die Temperatur einen kritischen Punkt, schaltet der Roboter
        sich selbstständig ab\footnote{Dabei verliert der Roboter seine
            Körperspannung und fällt um, was zu Schäden am Roboter führen
            kann.
        }.

        Um eine Notabschaltung am Roboter vorzunehmen kann der Hauptknopf
        (Brust) zwei mal schnell gedrückt werden.

    \section{Verbindung und Programmierung}

        Der NAO besitzt zwei eingebaute Computer (\autoref{img:nao:computers}),
        welche in Kopf und Brust
        untergebracht sind und verschiedene Aufgaben erfüllen.
        Sie werden benutzt um Eingaben und Ausgaben zu verarbeiten sowie um die
        Motoren des Roboters zu steuern.
        Es existiert eine Entwicklungsumgebung mit welcher diese Computer in
        verschiedenen Sprachen, wie zum Beispiel C, C++, Java und Python
        programmiert werden können.

        Um den NAO mit Verhaltensstrukturen (in weiterem "`Behaviour'') zu
        programmieren kann eine \ac{LAN}- oder \ac{Wifi}-Verbindung mit dem
        Roboter hergestellt werden.


