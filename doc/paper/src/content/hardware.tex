\chapter{Hardwareansteuerung}

    In folgendem Kapitel wird die Hardwareansteuerung des Roboters dokumentiert.

    \section{Behaviour}

        Um Behaviour, welche in dem Roboter abgespeichert sind, starten und
        stoppen zu können, wird zu Beginn des Programmes ein
        "`Behaviour-Objekt'' mit der \ac{IP}-Adresse als Parameter instanziiert
        (\autoref{lst:initbeh}).

\begin{lstlisting}[language=c++,
                   caption={Initialisierung Behaviour},
                   label={lst:initbeh}]
behaviour::Behaviour behaviourProxy(robotIp);
\end{lstlisting}

        Um später per Namespace auf die verschiedenen Behaviour zugereifen zu
        können ohne den Namen der Behaviour wissen zu müssen, werden die
        Behaviours mit einem Enumerator-Wert (\autoref{lst:behenum}) und deren
        richtigen Namen in einer Map gespeichert (\autoref{lst:behmap}).

\begin{lstlisting}[language=c++,
                   caption={Behaviour Enumerator},
                   label={lst:behenum}]
enum BehaviorID {
    Stand_up,
    Pickup,
    Throw,
    Register_Color,
    Search_Position,
    Won,
    Lost,
    Turn_Around,
    Greetings,
    ALL
};
\end{lstlisting}

        \autoref{lst:behenum} sicher zur Compiletime, dass die Behaviour richtig
        referenziert werden (Konzept des Single-Point-of-Failure).

        Um die Behaviour nun starten und stoppen zu können wurden zusätzliche
        Hilfsfunktionen erstellt
        (\autoref{lst:behstart}, \autoref{lst:behstop}).

    \section{Navigation}

        Um den Roboter steuern zu können, wird zu Beginn des Programmes ein
        Navigation-Objekt mit der \ac{IP}-Adresse als Parameter instanziiert
        (\autoref{lst:navinst}).

\begin{lstlisting}[language=c++,
                   caption={Navigation Instanziierung},
                   label={lst:navinst}]
navigation::Navigation navigationProxy(robotIp);
\end{lstlisting}

        Für die Steuerung wird in \autoref{lst:navmove} eine Funktion mit drei
        Parametern definiert.

\begin{lstlisting}[language=c++,
                   caption={Funktion: "`moveTo''},
                   label={lst:navmove}]
void Navigation::moveTo(const float x, const float y, const float rad) {
    globalNavProxy->moveTo(x, y, rad);
}
\end{lstlisting}

    \section{ImageLoader}

        Um ein Bild von dem Roboter auf den Steuer-Computer übertragen zu können
        wird zu Beginn des Programms ein "`ImageLoader''-Objekt instanziiert
        (\autoref{lst:imginst}).
        Dieses wird mit der \ac{IP}-Adresse und einer Pfadangabe zur Speicherung
        des Bildes instanziiert.

\begin{lstlisting}[language=c++,
                   caption={ImageLoader Instanziierung},
                   label={lst:imginst}]
imgloader::ImageLoader imageLoaderProxy(robotIp, pathToFile);
\end{lstlisting}

        Zu Beginn muss der VideoStream des NAO-Roboters "`angemeldet'' werden.
        Anschließend kann das Bild auf den Computer übertragen, und in ein
        Bild-Objekt von OpenCV gespeichert werden.
        Nach dem das Bild auf dem Computer gespeichert wurde, muss der
        VideoStream „abgemeldet“ und das Bild über die OpenCV-Funktion
        "`imwrite'' auf den Computer gespeichert werden.
        Das SDK bietet sechs unterschiedliche Bildqualitäten an, wie in
        \autoref{tbl:qualities} gelistet.

        \begin{table}[h]
            \caption{SDK Bildqualität}
            \label{tbl:qualities}
            \begin{center}
                \begin{tabular}[]{| l | l | l |}
                    AL::kQQQQVGA & 8 & Image of $40 * 30px$ \\
                    AL::kQQQVGA  & 7 & Image of $80 * 60px$ \\
                    AL::kQQVGA   & 0 & Image of $160 * 120px$ \\
                    AL::kQVGA    & 1 & Image of $320 * 240px$ \\
                    AL::kVGA     & 2 & Image of $640 * 480px$ \\
                    AL::k4VGA    & 3 & Image of $1280 * 960px$ \\
                \end{tabular}
            \end{center}
        \end{table}

\begin{lstlisting}[language=c++,
                   caption={Funktion: "`ImageLoader::getImage()''},
                   label={lst:imgget}]
void ImageLoader::getImage(){
    const std::string clientName = globalVideoProxy->subscribe("imgloader", AL::k4VGA, AL::kBGRColorSpace, 5);
    cv::Mat imgHeader = cv::Mat(cv::Size(1280, 960), CV_8UC3);
    AL::ALValue img = globalVideoProxy->getImageRemote(clientName);
    imgHeader.data = (uchar*) img[6].GetBinary();
    globalVideoProxy->releaseImage(clientName);
    globalVideoProxy->unsubscribe(clientName);
    cv::imwrite(pathToFile, imgHeader);
}
\end{lstlisting}

    \section{Main-Funktion}

        Um mit dem Roboter Sprachausgaben tätigen zu können, wird neben dem
        Behaviour-, Navigation- und ImageLoader-Objekt ein
        ALTextToSpeechProxy-Objekt instanziiert.
        Dieses wird im Anschluss mit Lautstärke und Sprache konfiguriert.

\begin{lstlisting}[language=c++,
                   caption={Instanziierung ALTextToSpeechProxy},
                   label={lst:alttts}]
AL::ALTextToSpeechProxy speakProxy(robotIp, 9559);
speakProxy.setLanguage("German");
speakProxy.setVolume(0.3);
behaviorProxy.startBehavior(behavior::Greetings);
speakProxy.say("Lass uns beginnen, ich fange an");
\end{lstlisting}

        Bevor der Spielzyklus beginnt, begrüßt der NAO-Roboter das Publikum mit
        dem Behavior „Greetings“ und teilt mit, dass er nun bereit ist
        anzufangen (\autoref{lst:alttts}).

        Der NAO-Roboter und somit auch der Spielzyklus beginnt damit, dass der
        Roboter den Würfel aufhebt und auf eine Eingabe wartet, damit der
        Roboter in die richtige Wurfposition positioniert werden kann.

        Nach dem eine Eingabe getätigt wurde, wirft der Roboter den Würfel.
        Nun begibt sich der Roboter in die Suchposition, sagt dass
        er nun mit der Suche des Würfels beginnt und schießt ein Foto.

        Zu Beginn sucht der Roboter in der Ausgangsposition nach dem Würfel.
        Wenn der Würfel nicht gefunden werden kann, dreht sich der Roboter um
        60 Grad nach links.
        Dort schießt er erneut ein Foto und führt seine Suche durch.
        Bei erneutem Fehlschlag dreht sich der Roboter um 120 Grad nach rechts,
        schießt ein Foto und führt seine Suche durch.
        Wenn der Würfel bis dahin noch nicht gefunden wurde, liegt der Würfel
        mit größter Wahrscheinlichkeit genau vor seinen Füßen, und somit
        außerhalb seines Sichtbereiches.
        Um dieses Problems aus dem Weg zu gehen, dreht sich der Roboter in die
        Ausgangsposition und läuft 20 Zentimeter nach hinten.
        Dort schießt er ein Bild und analysiert es.
        Wenn nun dort auch kein Würfel gefunden werden kann, wird das Programm
        mit einer Exception beendet.
        Bei erfolgreicher Suche werden die x-, y-Koordinate und der Winkel
        berechnet und in die lokalen Variablen gespeichert und eine
        Sprachausgabe getätigt.

        Nachdem der Würfel gefunden wurde, werden die x –und y-Koordinate von
        Meter in Zentimeter umgerechnet, und die x,-y-Koordinate und der Winkel
        dem Rutschen des Roboters angepasst.
        Wenn dann alle Werte richtig umgerechnet und angepasst sind, läuft der
        Roboter zuerst den kompletten Weg in y-Richtung und anschließend den
        halben Weg in x-Richtung.
        Der Grundgedanke hierbei ist, mögliche Abweichungen des Roboters durch
        Aufteilung des Laufweges in zwei Teile, zu kompensieren.

        Nun werden die zwei letzten Schritte (FindCube, WalkToCube) wiederholt
        nur mit dem kleinen Unterschied, dass bei „WalkToCube“ der x-Wert nicht
        halbiert, sondern vollständig gegangen wird.

        Jetzt befindet sich der Roboter vor dem Würfel, und signalisiert mit
        einer Sprachausgabe, dass dieser nun schaut welche Zahl er gewürfelt
        hat.
        Hierbei wird das Behavior„"`Register\_Color'' aufgerufen und
        anschließend ein Bild geschossen.
        Nun wird analysiert(Farberkennung – Ziel2) welche Zahl gewürfelt worden
        ist.
        Diese Zahl wird anschließend in den Spielalgorithmus eingespeist.
        Dann sagt der Roboter welche Zahl er gewürfelt hat - ab der zweiten
        Runde sagt der Roboter noch wie viel er insgesamt hat.
        Als nächstes wird per „algo.doDraw()“ abgefragt, ob der Roboter noch
        einmal würfeln soll.
        Wenn ja, sagt der Roboter, dass er noch einmal würfelt und der
        Spielzyklus beginnt von vorne.
        Wenn nein, wird der Spielzyklus verlassen.

        Anschließend sagt der Roboter dass er mit Würfeln fertig ist.
        Dann wird überprüft, ob der Roboter gewonnen (Behavior::Won), verloren
        (Behavior::Lost), oder ob der Gegenspieler(Bank) einmal würfeln darf.
        Wenn der Gegenspieler einmal würfeln darf, sagt der Roboter dies, und
        wartet bis eine Eingabe getätigt wird um würfeln zu können.
        Nachdem die Eingabe getätigt wurde beginnt der Roboter wie zuvor den
        Würfel zu suchen (FindCube, WalkToCube).
        Nachdem der Roboter den Würfel gefunden hat, sagt er dass er nun schaut
        welche Zahl der Gegenspieler gewürfelt hat.
        Dann wird überprüft welche Zahl gewürfelt wurde (Farberkennung – Ziel2).
        Anschließend sagt der Roboter welche Zahl er gewürfelt hat und es wird
        berechnet ob der Roboter, oder der Gegenspieler gewonnen hat.
        Wenn der Gegenspieler gewonnen hat wird das Behavior "`Lost'' ausgeführt
        und der Roboter sagt, dass der Gegenspieler gewonnen hat.
        Wenn der Roboter gewonnen hat, wird das Behavior „Won“ ausgeführt und
        der Main-Cycle ist beendet.
