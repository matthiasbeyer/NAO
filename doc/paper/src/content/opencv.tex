\chapter{Farberkennung}

    Die Implementierung der Farberkennung wurde in zwei Ziele unterteilt.
    \begin{itemize}
        \item Erkennung des Würfels (Rot)
        \item Erkennung der Augenzahl des Würfels (Gelb)
    \end{itemize}
    Generell wird dabei vom NAO zunächst ein Bild augenommen.
    Anhand dieses Bildes wird im nächsten Schritt die Erkennung durchgeführt.

    \section{Würfelerkennung}

        Im Falle der Erkennung des Würfels selbst wird über eine Farbanalyse
        nach der Farbe Rot gesucht und die Koordinaten des erkannten Bereichs
        mitsamt des Winkels zum erkannten diesem errechnet.
        Sollte das Programm mehrere Farbbereiche erkennen, werden diese
        gefiltert und nur der Größte wird im weiteren Programmverlauf verwendet.
        Somit werden Fehlerquellen minimiert.
        Die Koordinaten des erkannten Bereichs werden dem Programm zur Verfügung
        gestellt.

    \section{Erkennung der Augenzahl}

        Im Falle der Erkennung der Augenzahl des Würfels wird mit Hilfe der
        Farbanalyse nach gelben Farbsegmenten gesucht.
        Die so gefundenen Farbelemente werden nach ihrer Größe gefiltert und
        gezählt.
        Diese Vorgehensweise soll zur Fehlervermeidung beitragen, da der Würfel
        oft leicht schräg abgebildet wird, was dazu führt das nicht nur die
        Oberseite des Würfels auf der Abbildung sichtbar ist, sondern auch
        dessen Seiten und deren Augen.
        Diese werden bei beschriebenem Filtervorgang entfernt, da diese auf der
        Abbildung kleiner sind.
        Lediglich die Anzahl der so gefundenen Farbelemente wird dem Programm
        zur Verfügung gestellt.

        Letzteres wird mit einer \ac{HSV}-Analyse realisiert.
        Mittels dieser wird überprüft, ob gefundene Farbbereiche in der
        gesetzten \ac{HSV}-Reichweite liegen.
        Um später im Programm Berechnungen durchführen zu können, wird parallel
        zu diesen Werten der gefundene Farbwert, die Koordinaten des
        Mittelpunktes des Farbbereichs sowie die Größe des Farbbereichs in einer
        Liste abgelegt.

    \section{Implementierungsdetails}

        Die Implementierung der Augenzahl-Erkennung basiert auf
        \cite{github:motboc}.

        Es wurde ein Datentyp angelegt, welcher Informationen zu erkannten
        Farbbereichen gruppiert (\ref{lst:farbpos}).

\begin{lstlisting}[language=c++,
                   caption={Typ: "`FarbPos''},
                   label={lst:farbpos}]
struct FarbPos {
    std::string farbe;
    int         xPos;
    int         yPos;
    int         size;
    FarbPos(std::string c, int x, int y, int s):
        farbe(c), xPos(x), yPos(y), size(s)
    {
    }
}
\end{lstlisting}

        Für den in \autoref{lst:farbpos} aufgezigten Datentyp wurden folgende
        Operatoren überladen, um Verarbeitung der Werte zu vereinfachen:
        \begin{itemize}
            \item "`<''
            \item "`<=''
            \item "`>''
            \item "`>=''
            \item "`==''
            \item "`!=''
            \item "`int()''
            \item "`string()''
        \end{itemize}

        Ein Beispiel für den "`<''-Operator ist in
        \autoref{lst:farbpos:operator-lt} aufgezeigt.

\begin{lstlisting}[language=c++,
                   caption={Beispiel: Implementation "`FarbPos::operator<''},
                   label={lst:farbpos:operator-lt}]
friend bool
FarbPos::operator < (const FarbPos& self, const FarbPos& other)
{
    return self.yPos < other.yPos;
}
\end{lstlisting}

        Mit Hilfe der \ac{HSV}-Analyse wird festgestellt in welchem Bereich
        welcher \ac{HSV}-Wert liegt.
        Diese Werte werden anhand festgelegter Objekte mit einem Minimal- und
        einem Maximalwert verglichen.
        Damit kann errechnet werden, welcher \ac{HSV}-Wert welcher Farbe
        entspricht.
        Um weitere Rechnungen durchführen zu können, werden die Farben mit der
        dazugehörigen Koordinate des Mittelpunktes des Farbbereiches, sowie
        deren Größe, in einem sogenannten "`Rechenvektor'' gespeichert
        (\autoref{lst:savecolorpos}).

\begin{lstlisting}[language=c++,
                   caption={Abspeichern der Farbwerte},
                   label={lst:savecolorpos}]
void saveColorPos(std::string s, int x, int y, int size)
{
    auto fp = std::make_shared<FarbPos>(s, x, y, size);
    farblist.push_back(fp);
}
\end{lstlisting}

        \autoref{tbl:hsv} zeigt Werte, welche zur Farberkennung genutzt wurden.

        \begin{table}[h]
            \caption{Farbwerte}
            \label{tbl:hsv}
            \begin{center}
                \begin{tabular}[]{| l | l | l | l | l | l | l |}
                    &
                    \multicolumn{2}{|c|}{H} &
                    \multicolumn{2}{|c|}{S} &
                    \multicolumn{2}{|c|}{V} \\

                    Farbe
                    & min & max
                    & min & max
                    & min & max \\

                    \hline
                    yellow      & 10 & 30 & 170 & 220 & 205 & 255 \\
                    yellowdark  & 10 & 30 & 175 & 220 & 190 & 255 \\
                    yellowthree & 10 & 30 & 170 & 220 & 220 & 255 \\
                    red         & 0  & 10 & 140 & 255 & 70  & 255 \\
                    reddark     & 0  & 10 & 140 & 250 & 161 & 255 \\

                \end{tabular}
            \end{center}
        \end{table}

        Nachdem das größte Farbsegment (Würfel) gefiltert wurde, werden die
        Koordinaten gesondert abgespeichert.

\begin{lstlisting}[language=c++,
                   caption={Finden der größten Farbfläche},
                   label={lst:biggestcolorsurface}]
FarbPos getTopColor()
{
    if (!farblist.empty()) {
        result = *farblist.at(0);
        for (int i = 0; i < farblist.size(); i++) {
            if (farblist.at(i) -> size > result.size) {
                result = *farblist.at(i);
            }
        }
    }
    return result;
}
\end{lstlisting}

        Im Anschluss wird die obere Fläche gesucht
        (\autoref{label:biggestcolorsurface} und die errechneten Koordinaten des
        Bildes in Weltkoordinaten umgerechnet.

        \begin{equation} \label{eq:koordiform}
            E_{cm} = \frac{P - B}{k}
        \end{equation}

        Dabei ist $P$ die Anzahl der Pixel, $B$ die Basis, $k$ der Quotient und
        $E_{cm}$ das Ergebnis in Zentimeter.

        Hierfür wurde die Ausrichtung des NAO-Roboters eindeutig definiert und
        mittels eines Bildes auf welchem Markierungen abgelichtet wurden
        (wie in \ref{img:marks})
        die Quotienten berechnet, mittels welchen dann die Weltkoordinaten
        errechnet werden können (\ref{eq:koordiform}).
        Mit Hilfe dieser konnte \autoref{tbl:cmtbl} errechnet werden.

        \begin{equation} \label{eq:ykoord}
            E_{cm} = \frac{B - P}{k}
        \end{equation}

        Dabei ist $P$ die Anzahl der Pixel, $B$ die Basis, $k$ der Quotient und
        $E_{cm}$ das Ergebniss in Zentimeter.

        Da der Quotient $pixel / cm$ in jedem der Segmente $14$ beträgt, wurde
        keine Tabelle angefertigt.

        Die Funktion in \autoref{lst:getxy} kalkuliert, in welchem Segment der
        $y$-Wert liegt und errechnet mit Hilfe der Formel die
        $x$-(Welt-)Koordinate.
        Diese Werte werden dann per Übergabwert an die aufrufende Funktion
        zurückgegeben.

\begin{lstlisting}[language=c++,
                   caption={Funktion: "`getXY''},
                   label={lst:getxy}]
void getXY(float& x, float& y) {
    int start = 20; // in cm
    if(centerPoint.y >= 752) {
        x = start + (centerPoint.y - 752) / 20.8;
    }else if(centerPoint.y >= 615) {
        x = start + 10 + (centerPoint.y - 615) / 13.7;
    }else if(centerPoint.y >= 512) {
        x = start + 20 + (centerPoint.y - 512) / 10.3;
    }else if(centerPoint.y >= 416) {
        x = start + 30 + (centerPoint.y - 416) / 9.6;
    }else if(centerPoint.y >= 345) {
        x = start + 40 + (centerPoint.y - 345) / 7.1;
    }else if(centerPoint.y >= 285) {
        x = start + 50 + (centerPoint.y - 285) / 6;
    }else if(centerPoint.y >= 232) {
        x = start + 60 + (centerPoint.y - 232) / 5.3;
    }else if(centerPoint.y >= 189) {
        x = start + 70 + (centerPoint.y - 189) / 4.3;
    }else if(centerPoint.y >= 150) {
        x = start + 80 + (centerPoint.y - 150) / 3.9;
    }else if(centerPoint.y >= 117) {
        x = start + 90 + (centerPoint.y - 117) / 3.3;
    }
    y = (640 - centerPoint.x) / 14;
}
\end{lstlisting}

        Im letzten Schritt wird der Winkel zum gefundenen Würfel errechnet.
        Dazu wurde der Punkt "`centerPoint'' initialisiert, welcher nun
        verwendet wird.

\begin{lstlisting}[language=c++,
                   caption={Funktion: "`getAngle''},
                   label={lst:getAngle}]
void getAngle(cv::Point pt) {
    cv::Point center(width/2, height);
    cv::Point zero(0, 0);

    const double Rad2Deg = 180.0 / M_PI;

    //TODO: Change when first real start
    //return atan2((double) center.x - (double) pt.x, (double) center.y - (double) pt.y) * Rad2Deg; //degree
    return -atan2((double) center.x - (double) pt.x, (double) center.y - (double)pt.y); //rad
}
\end{lstlisting}

        Um die gefundenen Farbobjekte herausfiltern zu können, wird zunächst die
        Größe der Konturen überprüft.
        Nur Konturen mit einer Größe von $> 85$ werden selektiert.

\begin{lstlisting}[language=c++,
                   caption={Funktion: "`getDiceValue''},
                   label={lst:getDiveValue1}]
void getDiveValue() {
    int counter = 1;
    int diffx, diffy;
    std::vector<std::shared_ptr<FarbPos>> closerSelection;

    for(int i = 0; i < farblist.size(); i++){
        if(farblist.at(i)->size > 85){
            closerSelection.push_back(farblist.at(i));
        }
    }

    // ...
\end{lstlisting}

        Dann wird diese Selektion nach doppelt gefundenen Konturen durchsucht,
        da mehrere Gelbtöne erkannt werden müssen.
        Hierbei wird die Selektion nach Konturen durchsucht, welche in $x$- wie
        auch in $y$-Reichtung $<= 40 px$ beieinander liegen.
        Wird eine doppelte Kontur gefunden, wird diese nicht berücksichtigt
        und es wird die nächste Kontur selektiert.
        Die so erkannte Augenzahl wird zurückgegeben.

\begin{lstlisting}[language=c++,
                   caption={Funktion: "`getDiceValue''},
                   label={lst:getDiveValue2}]
    // ...
        counter = closerSelection.size();
    for(int i = 0; i < closerSelection.size() - 1; i++){
        for(int i2 = i + 1; i2 < closerSelection.size(); i2++){
            diffx = std::abs(closerSelection.at(i)->xPos - closerSelection.at(i2)->xPos);
            diffy = std::abs(closerSelection.at(i)->yPos - closerSelection.at(i2)->yPos);


            std::cout << "DIFF x: " << diffx << " y: " << diffy << std::endl;
            if(diffx <= 40 && diffy <= 40){
                counter--;
                break;
            }
        }
    }
    return counter;
}
\end{lstlisting}
