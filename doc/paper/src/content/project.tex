\chapter{Projekt}

    In diesem Kapitel soll beschrieben werden wie das Projekt aufgebaut und in
    Angriff genommen wurde.

    Die relevanten Rollen sowie deren Verteilung, der Projektablauf, die
    Schwierigkeiten sowie die gewonnenen Erfahrungen werden in diesem Kapitel
    umschrieben, können jedoch in nachfolgenden Kapitel noch vertieft werden.

    \section{Rollen}
        \label{chap:rollen}

        Im Semesterprojekt wurden folgende Rollen festgelegt:

        \begin{itemize}
            \item Entwickler
                \begin{itemize}
                    \item Ansteuerung Hardware \\
                        Schreibt oder Produziert Quellcode welcher mit der
                        Hardware des Roboters in direkter Verbindung steht.

                    \item Bilderkennung \\
                        Schreibt oder Produziert Quellcode welcher benutzt
                        werden kann um Daten aus Bildern zu extrahieren welche
                        für den Ablauf des Roboterprogramms nötig sind.

                    \item Quellcode \\
                        Quellcodeentwickler welcher Quellcode produziert welcher
                        nicht in die anderen Rollenbeschreibungen der Entwickler
                        passt.

                    \item Spielealgorithmus \\
                        Entwickelt und Implementiert den Spielealgorithmus.

                \end{itemize}
            \item Ersteller der Dokumentation \\
                Erstellt die Dokumentation/das Tutorial basierend auf Inhalten
                welche die anderen Projektmitglieder liefern.

            \item Leitung Marketing \\
                Entwirft und Erstellt Werbung und Marketinginhalte welche zur
                Abschlusspräsentation des Projektes verwendet werden.

            \item Projektleiter \\
                Leitet das Projekt, verwaltet Meilensteine und Arbeitspakete.
                Kümmert sich um eventuell benötigte Zusatzkomponenten für das
                Projekt.

            \item Protokollant \\
                Protokolliert Projekttreffen und notiert kritische
                Informationen, sofern notwendig.

            \item Präsentator \\
                Präsentiert das Projekt dem Kunden und vertritt dabei die
                Projektgruppe.

            \item Tester \\
                Testet das Produkt.

        \end{itemize}

        Jede Rolle sollte zwei Mal vergeben werden: Einmal an eine Hauptperson
        für diese Rolle, zudem sollte für jede Rolle ein Vertreter existieren.

    \section{Rollenverteilung}

        Die in \autoref{chap:rollen} aufgezeigten Rollen wurden wie Folgt
        verteilt:

        \begin{center}
            \begin{tabular}{l | l | l }
                Rollenname                       & Hauptperson     & Vertreter \\

                \hline

                Entwickler: Ansteuerung Hardware & Sardor Tarik    & Butrint Veselaj \\
                Entwickler: Bilderkennung        & Matthias Kotz   & Daniel Eisenreich \\
                Entwickler: Quellcode            & Alle            & Alle \\
                Entwickler: Spielealgorithmus    & Matthias Beyer  & Daniel Eisenreich \\
                Ersteller der Dokumentation      & Matthias Beyer  & Butrint Veselaj \\
                Leitung Marketing                & Sardor Tarik    & Matthias Kotz, Butrint Veselaj \\
                Projektleiter                    & Butrint Veselaj & \\
                Protokollant                     & Matthias Beyer  & Butrint Veselaj \\
                Präsentator                      & Matthias Kotz   & Daniel Eisenreich \\
                Tester                           & Alle            & Alle \\
            \end{tabular}
            \captionof{table}{Rollenverteilung}
            \label{tab:rollenverteilung}
        \end{center}

        Wie in \autoref{tab:rollenverteilung} zu sehen, wurde jedes
        Projektmitglied mit mehreren Rollen ausgestattet.
        Manche Rollen wurden von allen Projektmitgliedern belegt.

    \section{Projektablaufplanung}

        \notAvailable{}

    \section{Herausforderungen}

        \notAvailable{}

    \section{Gewonnene Erfahrungen}

        \notAvailable{}

