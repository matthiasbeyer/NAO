\chapter{Einführung}

    Im Rahmen des Semesterprojektes, welches im Wintersemester 2015/2016 an der
    Hochschule Furtwangen von den Projektmitgliedern absolviert wurde, wurde
    eine Software für einen NAO-Roboter entwickelt.
    Der NAO-Roboter ist ein seit 2008 von Aldebaran Robotics entwickelter
    Roboter welcher meist für akademische Zwecke eingesetzt wird.
    Die Software, welche für den Roboter entwickelt wurde, sollte nach den an
    der Hochschule gelehrten Prinzipien entwickelt und umgesetzt werden und den
    Roboter ein Spiel spielen lassen.
    Dieses Spiel musste von den Projektmitgliedern definiert und entwickelt
    werden.

    Die Zielsetzung des Projektes war die Dokumentation des Projektes sowie das
    Erstellen einer Anleitung für Nachfolgeprojekte.

    Diese Arbeit beinhaltet die Dokumentation und soll zugleich als
    Anleitung/Tutorial dienen.

