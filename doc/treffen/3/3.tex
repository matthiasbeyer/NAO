%%Standarddokument, v1.87 %%

\documentclass[a4paper,DIV=9,12pt]{scrreprt}
\usepackage[T1]{fontenc}
\usepackage[utf8]{inputenc}
\usepackage[ngerman]{babel}
\usepackage[colorlinks=false,pdfborder={0 0 0},bookmarksnumbered]{hyperref}
\usepackage{blindtext}
\usepackage{setspace}
% \usepackage{lmodern}
% \usepackage{libertine}
\usepackage{microtype}

\usepackage{etoolbox}
\makeatletter
\patchcmd{\chapter}{\if@openright\cleardoublepage\else\clearpage\fi}{}{}{}
\makeatother

\begin{document}

%\parindent0mm

\hypersetup{
	pdftitle={Handout},
	pdfauthor={},
    pdfsubject={Projekttreffen Nr. 3}
	}

\onehalfspace

\begin{titlepage}
    \title{Handout}
    \subtitle{Projekttreffen Nr. 3}
    \author{}
    \date{21-10-2015}
    \maketitle
    \thispagestyle{empty}
\end{titlepage}
\setcounter{page}{1}

%%%%%%%%%%%%%%%%%%%%%%%%%%%%%%%%%%%%%%%%%%%%%%%%%%%%%%%%%%%%%%%%%%%

\chapter{NAO-Setup}

Das Setup des NAO wurde mit Frau Jakob besprochen und ausprobiert.
Den NAO aufstehen und hinsetzen zu lassen hat bei einem ersten Test
funktioniert, allerdings war die Verbindung mittels WLAN äusserst schlecht und
der physische Roboter hat erst nach mehreren Minuten auf gestartete Aktionen
reagiert.

Nachdem diese Schwierigkeiten überwunden wurden, konnten wir mehrere
Testprogramme auf dem NAO ausführen.

Von Frau Jakob erhielten wir zudem den Quellcode der Vorgängergruppen.
Diesen Einblick erachten wir als besonders wichtig, da er einen ersten Kontakt
mit der Materie im Sinne der Projekterstellung bezüglich der Aspekte
Programmierung und Softwaredesign darstellte.

Es war für Daniel nicht möglich, das ``qibuild''-Werkzeug zu installieren.
Zudem war das Einrichten des Netzwerks im Labor war für Matthias schwierig.

Zuletzt war es jedoch Möglich den Beispielquellcode, welcher bei der
NAO-Installations-CD mitgeliefert wurde, zu übersetzen.

Zudem wurde ein Virtualbox-Disc-Image angelegt.

\end{document}
